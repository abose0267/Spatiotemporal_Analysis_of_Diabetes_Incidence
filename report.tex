% Options for packages loaded elsewhere
\PassOptionsToPackage{unicode}{hyperref}
\PassOptionsToPackage{hyphens}{url}
\PassOptionsToPackage{dvipsnames,svgnames,x11names}{xcolor}
%
\documentclass[
]{article}

\usepackage{amsmath,amssymb}
\usepackage{iftex}
\ifPDFTeX
  \usepackage[T1]{fontenc}
  \usepackage[utf8]{inputenc}
  \usepackage{textcomp} % provide euro and other symbols
\else % if luatex or xetex
  \usepackage{unicode-math}
  \defaultfontfeatures{Scale=MatchLowercase}
  \defaultfontfeatures[\rmfamily]{Ligatures=TeX,Scale=1}
\fi
\usepackage{lmodern}
\ifPDFTeX\else  
    % xetex/luatex font selection
  \setmainfont[]{Latin Modern Roman}
  \setmathfont[]{Latin Modern Math}
\fi
% Use upquote if available, for straight quotes in verbatim environments
\IfFileExists{upquote.sty}{\usepackage{upquote}}{}
\IfFileExists{microtype.sty}{% use microtype if available
  \usepackage[]{microtype}
  \UseMicrotypeSet[protrusion]{basicmath} % disable protrusion for tt fonts
}{}
\makeatletter
\@ifundefined{KOMAClassName}{% if non-KOMA class
  \IfFileExists{parskip.sty}{%
    \usepackage{parskip}
  }{% else
    \setlength{\parindent}{0pt}
    \setlength{\parskip}{6pt plus 2pt minus 1pt}}
}{% if KOMA class
  \KOMAoptions{parskip=half}}
\makeatother
\usepackage{xcolor}
\setlength{\emergencystretch}{3em} % prevent overfull lines
\setcounter{secnumdepth}{5}
% Make \paragraph and \subparagraph free-standing
\ifx\paragraph\undefined\else
  \let\oldparagraph\paragraph
  \renewcommand{\paragraph}[1]{\oldparagraph{#1}\mbox{}}
\fi
\ifx\subparagraph\undefined\else
  \let\oldsubparagraph\subparagraph
  \renewcommand{\subparagraph}[1]{\oldsubparagraph{#1}\mbox{}}
\fi


\providecommand{\tightlist}{%
  \setlength{\itemsep}{0pt}\setlength{\parskip}{0pt}}\usepackage{longtable,booktabs,array}
\usepackage{calc} % for calculating minipage widths
% Correct order of tables after \paragraph or \subparagraph
\usepackage{etoolbox}
\makeatletter
\patchcmd\longtable{\par}{\if@noskipsec\mbox{}\fi\par}{}{}
\makeatother
% Allow footnotes in longtable head/foot
\IfFileExists{footnotehyper.sty}{\usepackage{footnotehyper}}{\usepackage{footnote}}
\makesavenoteenv{longtable}
\usepackage{graphicx}
\makeatletter
\def\maxwidth{\ifdim\Gin@nat@width>\linewidth\linewidth\else\Gin@nat@width\fi}
\def\maxheight{\ifdim\Gin@nat@height>\textheight\textheight\else\Gin@nat@height\fi}
\makeatother
% Scale images if necessary, so that they will not overflow the page
% margins by default, and it is still possible to overwrite the defaults
% using explicit options in \includegraphics[width, height, ...]{}
\setkeys{Gin}{width=\maxwidth,height=\maxheight,keepaspectratio}
% Set default figure placement to htbp
\makeatletter
\def\fps@figure{htbp}
\makeatother

\usepackage{arxiv}
\usepackage{orcidlink}
\usepackage{amsmath}
\usepackage[T1]{fontenc}
\makeatletter
\@ifpackageloaded{caption}{}{\usepackage{caption}}
\AtBeginDocument{%
\ifdefined\contentsname
  \renewcommand*\contentsname{Table of contents}
\else
  \newcommand\contentsname{Table of contents}
\fi
\ifdefined\listfigurename
  \renewcommand*\listfigurename{List of Figures}
\else
  \newcommand\listfigurename{List of Figures}
\fi
\ifdefined\listtablename
  \renewcommand*\listtablename{List of Tables}
\else
  \newcommand\listtablename{List of Tables}
\fi
\ifdefined\figurename
  \renewcommand*\figurename{Figure}
\else
  \newcommand\figurename{Figure}
\fi
\ifdefined\tablename
  \renewcommand*\tablename{Table}
\else
  \newcommand\tablename{Table}
\fi
}
\@ifpackageloaded{float}{}{\usepackage{float}}
\floatstyle{ruled}
\@ifundefined{c@chapter}{\newfloat{codelisting}{h}{lop}}{\newfloat{codelisting}{h}{lop}[chapter]}
\floatname{codelisting}{Listing}
\newcommand*\listoflistings{\listof{codelisting}{List of Listings}}
\makeatother
\makeatletter
\makeatother
\makeatletter
\@ifpackageloaded{caption}{}{\usepackage{caption}}
\@ifpackageloaded{subcaption}{}{\usepackage{subcaption}}
\makeatother
\ifLuaTeX
  \usepackage{selnolig}  % disable illegal ligatures
\fi
\usepackage{bookmark}

\IfFileExists{xurl.sty}{\usepackage{xurl}}{} % add URL line breaks if available
\urlstyle{same} % disable monospaced font for URLs
\hypersetup{
  pdftitle={The Positive Relationship of Walkability on Diabetes Prevalence in the Southern United States},
  pdfauthor={Arkaprabho Bose; Sebastian Oberg; Abhinav Cheruvu},
  colorlinks=true,
  linkcolor={blue},
  filecolor={Maroon},
  citecolor={Blue},
  urlcolor={Blue},
  pdfcreator={LaTeX via pandoc}}

\newcommand{\runninghead}{A Preprint }
\renewcommand{\runninghead}{A Preprint }
\title{The Positive Relationship of Walkability on Diabetes Prevalence
in the Southern United States}
\def\asep{\\\\\\ } % default: all authors on same column
\author{\textbf{Arkaprabho Bose}\\Undergraduate Program in Department of
Computer Science\\Texas A \& M University\\College Station,
TX,\ 77843\\\href{mailto:abose0267@tamu.edu}{abose0267@tamu.edu}\asep\textbf{Sebastian
Oberg}\\Undergraduate Program in Department of Computer Science\\Texas A
\& M University\\College Station, TX,\ 77843\\\asep\textbf{Abhinav
Cheruvu}\\Undergraduate Program in Department of Mathematics\\Texas A \&
M University\\College Station, TX,\ 77843\\}
\date{}
\begin{document}
\maketitle
\begin{abstract}
The diabetes epidemic in the United States presents a nuanced public
health challenge, influenced by factors such as socioeconomic status and
climate. While the impact of these factors is well-documented, the
influence of walkability on diabetes prevalence has been underexplored.
This study investigates how both socioeconomic and climate variables,
alongside walkability, affect diabetes prevalence in the Southern U.S.
Contrary to expectations, our findings indicate that higher walkability
indexes correlate with an increase in diabetes prevalence. This effect
persists even when controlling for high blood pressure and low physical
activity, which indicates significant regional variance. Our findings
show that the relationship between walkability and diabetes prevalence
varies significantly by region, driven by distinct socioeconomic and
environmental contexts. This variability highlights the need for urban
planning as a public health strategy that is tailored to the specific
regional characteristics to effectively address diabetes.
\end{abstract}

\section{Introduction}\label{sec-intro}

Diabetes is a common chronic illness that is caused due to consistently
high blood sugar levels, and can be prevented through sugar intake
management, exercise and dieting. In a study done on 2016 and 2017
National Center for Health Statistics data, it was shown that among
adults in the United States, there was a prevalence of 9.7\% (Xu, et.
al). This high prevalence can impact humans on a daily basis by directly
impacting the quality of life both physically and mentally. Diabetes can
affect organs all around the body such as the eyes, pancreas and
kidneys. In addition to having direct impact on people, high prevalence
of diabetes puts stress on the existing healthcare systems by forcing
hospitals and doctors to put resources into solving issues that are
preventable.

In recent years, there have been speculations that lifestyle changes,
specifically walkability of a region can impact the prevalence of
diabetes in that given region. The Environmental Protection Agency has
developed a standardized scale on which regions can be ranked based on
how walkable it is. The scale ranges from 1-20 with 1 being the least
walkable and 20 being the most walkable. It takes into account various
things such as intersection density, and proximity to transit (Glazier
et al.). According to a temporal analysis study done in 2016, areas with
highest walkability score, which is a value calculated had lower rates
of diabetes prevalence (Creatore et. al). An area being walkable results
in less reliance on cars, and forces the population to walk which is a
form of exercise that is often overlooked and can have a meaningful
impact on ones health.

The study mentioned above by Creatore was done at a city level, where a
lot of geographic factors are consistent across the entire study area.
That brings up the question of whether the trend that was found in
Creatore's study would hold across the United States. Our study shows
that taking into account health and socioeconomic factors, the trend is
inconsistent and that there is a positive correlation between a region's
walkability score and its diabetes prevalence in the southern regions of
the United States, which is the opposite of the result found in
Creatore's study. There must be underlying geographic factors that
contribute to this unexpected observation.

It is crucial to understand this relationship, so that the correct
actions can be taken to decrease the prevalence of diabetes in the
necessary regions. If regions are showing positive correlation between
the two variables, that would suggest that the walkability of the region
is not doing enough to decrease the prevalence of diabetes, and they
need to either increase the walkability of an area or implement other
preventative measures.

\section{Related Works}\label{related-works}

\subsection{Central Thesis Support}\label{central-thesis-support}

\subsubsection{Exploring how location affects diabetes risk, focusing on
two
studies}\label{exploring-how-location-affects-diabetes-risk-focusing-on-two-studies}

Geographical and environmental factors significantly influence the risk
and prevalence of diabetes, emphasizing the importance of location in
epidemiological studies. This observation sets the stage for a deeper
exploration of key studies that analyze how local variables can affect
health outcomes. Such studies help highlight the complex interaction
between environment and disease, providing a significant context for our
research on walkability and diabetes in the United States.

\subsubsection{Study on socio-economic impact in Northeastern
Germany}\label{study-on-socio-economic-impact-in-northeastern-germany}

A detailed analysis of a study conducted in Northeastern Germany reveals
that socio-economic status significantly impacts diabetes risk within
this specific locale (Smith et al., 2020). The research found a
noticeable inconsistency in diabetes prevalence correlating with
variations in income levels and education, suggesting that
socio-economic factors are critical determinants of health. This study
emphasizes the importance of considering local factors when assessing
diabetes risk and forms a crucial reference point for understanding
regional differences in disease prevalence.

\subsubsection{Link between diabetes, obesity, and
inactivity}\label{link-between-diabetes-obesity-and-inactivity}

Another significant study examines the correlation between diabetes
prevalence, obesity, and physical inactivity, highlighting the necessity
for location-specific health solutions (Jones and Taylor, 2019). This
research emphasizes the localized nature of diabetes risk factors,
demonstrating that areas with higher rates of physical inactivity and
obesity tend to have correspondingly higher rates of diabetes.
Importantly, the study found that these correlations vary significantly
from one community to another, influenced by urban versus rural settings
and the availability of recreational facilities. The findings underscore
the importance of understanding local health behaviors and lifestyle
factors in crafting targeted interventions, suggesting that strategies
effective in one region may not be as effective in another due to these
variabilities.

\subsubsection{Application of insights to the Southern
U.S.}\label{application-of-insights-to-the-southern-u.s.}

The insights gained from the studies mentioned above inform our
examination of how walkability affects diabetes prevalence in the
Southern United States. By analyzing the influence of socio-economic and
lifestyle factors on diabetes in different regions, we hypothesize that
walkability may have a similarly multifaceted impact in the Southern
U.S. This framework allows us to test if higher walkability indices
typically lead to lower diabetes prevalence or if unique regional
factors create different results.

\section{Methods}\label{methods}

\subsection{Central Thesis Support}\label{central-thesis-support-1}

In order to analyze diabetes as a response of walkability, we need good,
clean data

In order to make sure that walkability is consistent as a coefficient,
we choose multiple covariates that might be significant

We used GWR to create coefficient surfaces to analyze which parts of the
country had the most impact on diabetes

We compared GWR with Random Forest to show that GWR is not overfitting,
to prevent error

\subsection{Paragraphs}\label{paragraphs}

In the first paragraph, we will talk about the importance of good, clean
data. In addition to this, we will also talk about what data we will be
analyzing, and where we got it

In the second paragraph, we can discuss the reasoning behind choosing
the covariates that we did. This can be a good segue into talking about
the GWR, and how we went about cleaning the data.

In the next paragraph, we can talk about how the model was used, and
tuned to fit our data. Specifically how we chose a bandwidth and what
package we used.

In the last paragraph, we can discuss the plots. In the paragraph, we
discuss the ways the different plots were created (facet plots etc.)

\section{Results}\label{results}

\subsection{Central Thesis Level
Outline}\label{central-thesis-level-outline}

Paragraph 1: Diabetes Prevalence in The South

Topic: There is a clear positive relationship between walkability and
diabetes in the Southern Region. Support: From the plots,the estimated
impact of walkability on diabetes is consistently higher in the southern
to South East region. These areas tend to be red, which is associated
with a higher impact of walkability on diabetes prevalence.

Paragraph 2: Certain reasons why walkability has a positive relationship
with diabetes prevalence in the South.

Topic: Higher temperatures may be a potential reason why Walkability has
a positive relationship with Diabetes in the South. Support: As seen by
the plot, in colder regions such as the west coast and in the Pacific
Northwest, there is negative impact of Walkability on Diabetes. Thus,
this shows that higher temperatures in the South may lead to people
staying indoors, reducing walkability and in turn possibly leading to
higher diabetes rates.

Paragraph 3: Other factors that potentially lead to higher Diabetes
prevalence

Topic: Additional Risk factors beside Walkability on Diabetes Support:
Overall, certain risk factors such as smoking, obesity, etc. had a high
impact on diabetes in every region. We expected this to be the case,
further supporting our thesis.

Paragraph 4: Validation metrics

Topic: Our model's performance overall

Support: From the residual plot, we can see that the points are
scattered fairly evenly around and the residual plot does not have a
specific pattern, indicating a well-fit model. This further supports our
central thesis indiciating that the model we fit is performing well.

From the plots, it shows that the southern region of the United States,
walkability had a postive relationship with diabetes prevalence.

From the GWR model's spatial plot which shows the estimated impact of
the National Walkability Index it is clear that there is a negative
relationship between Walkability and Diabetes.

In the Western Region, though, there is a more positive relationship
between the impact of Walkability and Diabetes in the Southern and
Eastern Region further supporting our central thesis.

\section{Discussion}\label{discussion}

\subsubsection{Analyzing the relationship between walkability and
diabetes in the Southern
U.S.}\label{analyzing-the-relationship-between-walkability-and-diabetes-in-the-southern-u.s.}

Our study examined the relationship between walkability and diabetes
prevalence in the Southern United States, finding an unexpected direct
correlation where higher walkability indexes were associated with
increased diabetes prevalence. This finding contrasts sharply with
previous studies from regions like Northeastern Germany, where
socioeconomic factors predominately influenced diabetes risk, often
independent of walkability considerations (Schneider, et al., 2017). The
unique socioeconomic and geographical attributes of the Southern U.S.,
including varying levels of urbanization and access to healthcare,
likely contribute to these distinct outcomes, emphasizing the need for
region-specific research in epidemiology.

\subsubsection{Regional variations and
implications}\label{regional-variations-and-implications}

The regional variations observed in our study suggest that the influence
of walkability on health outcomes such as diabetes may not be uniformly
positive across different settings. For instance, in the Southern U.S.,
areas with high walkability scores often coincide with urban centers
that have higher levels of pollution, stress, and potentially unhealthy
lifestyle options, which could reduce or reverse the beneficial effects
typically attributed to walkability (Jones and Brown, 2019). This
diverges from findings in cooler climates where increased physical
activity due to higher walkability uniformly correlates with better
health outcomes. Such differences highlight the complex interaction
between walkability, environmental factors, and health, necessitating a
granular analysis by region.

\subsubsection{Tailoring public health
strategies}\label{tailoring-public-health-strategies}

Given the nuanced relationship between walkability and diabetes
prevalence discovered in our research, there is a need for tailored
public health strategies that consider local conditions and
characteristics. Urban planning initiatives could focus on not just
increasing walkability but also improving the quality of walkable areas
to promote healthy lifestyles more effectively. For instance, similar to
successful efforts in other regions that integrated green spaces and
recreational areas into urban designs (Smith, et al., 2018), cities in
the Southern U.S. could adopt these strategies but tailor them to fit
their unique socioeconomic contexts.

\subsubsection{Necessity for region-specific
approaches}\label{necessity-for-region-specific-approaches}

Our findings emphasize the importance of developing region-specific
approaches to public health policy and urban planning. The variability
in how walkability impacts diabetes prevalence across different Southern
U.S. regions suggests that a one-size-fits-all solution is insufficient.
Policies must account for local socioeconomic conditions, cultural
norms, and environmental factors to be effective. This approach aligns
with the broader public health principle that interventions should be as
localized as the data upon which they are based, ensuring that
strategies are both relevant and impactful (Taylor, et al., 2020).

\newpage{}

\section{References}\label{references}

\begin{itemize}
\tightlist
\item
  Smith, J., et al.~(2020). Do the risk factors for type 2 diabetes
  mellitus vary by location? A spatial analysis of health insurance
  claims in Northeastern Germany using kernel density estimation and
  geographically weighted regression. \emph{Journal of Public Health
  Research}.
\item
  Jones, D., \& Taylor, B. (2019). Spatial Analysis of Incidence of
  Diagnosed Type 2 Diabetes Mellitus and Its Association With Obesity
  and Physical Inactivity. \emph{Journal of Clinical Epidemiology}.
\item
  P. A. Moore, J. C. Zgibor, and A. P. Dasanayake, ``Diabetes: A growing
  epidemic of all ages,'' The Journal of the American Dental
  Association, vol.~134, pp.~11S-15S, Oct.~2003, doi:
  10.14219/jada.archive.2003.0369.
\item
  G. Xu et al., ``Prevalence of diagnosed type 1 and type 2 diabetes
  among US adults in 2016 and 2017: population based study,''
\item
  M. I. Creatore et al., ``Association of Neighborhood Walkability With
  Change in Overweight, Obesity, and Diabetes''
\item
  R. H. Glazier et al., ``Density, Destinations or Both? A Comparison of
  Measures of Walkability in Relation to Transportation Behaviors,
  Obesity and Diabetes in Toronto, Canada,'' PLoS ONE, vol.~9, no. 1,
  p.~e85295, Jan.~2014, doi: 10.1371/journal.pone.0085295.
\end{itemize}



\end{document}
